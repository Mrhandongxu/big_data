\documentclass[a4paper]{article}
\usepackage{geometry}
\usepackage{longtable}
\usepackage{graphicx}
\usepackage{xcolor}
\usepackage{listings}
\usepackage[colorlinks]{hyperref}
\lstset{columns=fixed,             
        numbers=left, % 在左侧显示行号
        frame=none, % 不显示背景边框
        backgroundcolor=\color[RGB]{245,245,244}, % 设定背景颜色
        keywordstyle=\color[RGB]{40,60,255}, % 设定关键字颜色
        numberstyle=\footnotesize\color{darkgray}, % 设定行号格式
        commentstyle=\it\color[RGB]{100,100,100}, % 设置代码注释的格式
        stringstyle=\rmfamily\slshape\color[RGB]{128,0,0}, % 设置字符串格式
        showstringspaces=false, % 不显示字符串中的空格
        language=JAVA,
        breaklines=true,
        postbreak=\space,
        breakautoindent=true,
        breakindent=20pt,
}
\usepackage{float}
\geometry{left=2cm,right=2cm,top=1.5cm,bottom=2.5cm}
\title{Introduction to Big Data: Homework 1}
\author{Dongxu Han}
\date{}
\begin{document}
\maketitle
\setlength{\parskip}{-0.5em}
\Large
\noindent
1.\\
\textbf{title.akas.tsv} - this file describes details of    localized title/tv in different regions.
\begin{longtable}[l]{p{2.5cm}|p{10.5cm}}
\hline
Content & Description \\ \hline
titleId & unique id of title \\ \hline
ordering & unique id of each title in different regions \\ \hline
title & a localized title \\ \hline
region & display region of this version, represented by acronym of countries \\ \hline
language & language of this version, usually accords with region \\ \hline
types & display type of this version, like dvd, tv, imdbDisplay, etc. \\ \hline
attributes & more description of this version \\ \hline
isOriginal-Title & whether it's original title. 0 means not original; 1 means original \\ \hline
\end{longtable}
\noindent
\textbf{name.basics.tsv} - describe basic information of persons, including name, profession, etc.
\begin{longtable}[l]{p{2.5cm}|p{10.5cm}}
\hline
Content & Description \\ \hline
nconst & unique id of the person \\ \hline
primary-Name & name of the person \\ \hline
birthYear & birth year of the person \\ \hline
deathYear & death year of the person, may be null because being alive \\ \hline
primary-Profession & profession of the person \\ \hline
knownFor-Titles & titles this person is known for \\ \hline
\end{longtable}
\noindent
\textbf{title.basics.tsv} - describe basic information of titles, including type, genre, etc.
\begin{longtable}[l]{p{2.5cm}|p{10.5cm}}
\hline
Content & Description \\ \hline
tconst & unique id of a given title \\ \hline
titleType & type of a given title \\ \hline
primary-Title & the most used title \\ \hline
isAdult & adult restriction. 0-not adult; 1-adult \\ \hline
startYear & year this tile starts \\ \hline
endYear & year this title ends \\ \hline
runtime-Minutes & how much minutes this title runs \\ \hline
genres & genre of this title, like documentary, animation, etc. \\ \hline
\end{longtable}
\noindent
\textbf{title.crew.tsv} - describe directing and writing relationship of between titles and persons.
\begin{longtable}[l]{p{2.5cm}|p{10.5cm}}
\hline
Content & Description \\ \hline
tconst & unique id of a given title \\ \hline
directors & ids indicating the person who direct this title \\ \hline
writers & id indicating the person who write this title \\ \hline
\end{longtable}
\noindent
\textbf{title.episode.tsv} - describe episode information of titles and parenting relationship among titles.
\begin{longtable}[l]{p{2.5cm}|p{10.5cm}}
\hline
Content & Description \\ \hline
tconst & unique id of a given title \\ \hline
parent-Tconst & id indicating the parent title of a given title \\ \hline
season-Number & number of season which this title's episode belongs to \\ \hline
episode-Number & number of episode of this title \\ \hline
\end{longtable}
\noindent
\textbf{title.principals.tsv} - describe principal relationship between titles and persons and details of principals.
\begin{longtable}[l]{p{2.5cm}|p{10.5cm}}
\hline
Content & Description \\ \hline
tconst & unique id of a given title \\ \hline
ordering & a unique number of the principal in a given title \\ \hline
nconst & id indicating the person who is one of principals of a given title \\ \hline
category & the category of job the person is in for a given title \\ \hline
job & specific job the person does, may be $"\backslash n"$ \\ \hline
characters & name of the role this person plays, may be $"\backslash n"$ \\ \hline
\end{longtable}
\noindent
\textbf{title.ratings.tsv} - describe rating and voting information of titles.
\begin{longtable}[l]{p{2.5cm}|p{10.5cm}}
\hline
Content & Description \\ \hline
tconst & unique id of a given title \\ \hline
average-Rating & average rating of a given title \\ \hline
numVotes & number of received votes \\ \hline
\end{longtable}
\par
\noindent
2.\\\,ER diagram is represented in Figure 1.
\begin{figure}[ht]
\centering
\includegraphics[scale=0.26]{ER.png}
\caption{Entity-Relationship diagram.}
\label{fig:label}
\end{figure}
\\
\par
\noindent
3.\\
The \textbf{relational model} is represented in \textbf{Figure 2}.\\
To use integer instead of strings for primary keys, as for an alphanumeric id, I store the numeric part of it in database. When querying, we join the alphabetic head with it according to the circumstance.\\
Consider "tt000001". Split "tt000001" to "tt" and "000001" and store "000001" only. If we want to export the data in database, we may then join "tt" with the six-digit number.
\begin{figure}[ht]
\centering
\includegraphics[scale=0.5]{RM.png}
\caption{Relational model.}
\label{fig:label}
\end{figure}\\
Sql script to create a database for imdb is in file \href{run:./imdb\_script.sql}{imdb\_script.sql}.
\\\\
\par
\noindent
4.\\
As for my program, I try to create multiple threads to read different file and insert data into database simultaneously. As shown in relational model, I use a lot of "conceptual" foreign keys. The word "conceptual" means I don't physically use it while creating database and loading data. I admit that foreign key could help keep integrity of data. However, since database would make a check on foreign key when doing operations, this would reduce the performance of loading data into database. This doesn't mean I don't take care of integrity of data. It's just a strategy to improve performance by sacrificing other aspects. Besides, it is actually developer-friendly with no foreign key. We are able to check data integrity in our program as well, with no dependence on database itself.\\
Time reporting:\\
Thread loads ratings within 5.46 minutes.\\
Thread loads write and direct within 28.38 minutes.\\
Thread loads movie within 30.52 minutes.\\
Thread loads person within 38.37 minutes.\\
Thread loads act and produce within 103.35 minutes.\\
Finish loading all data within 103.36 minutes.\\\\
Program source code: \href{run:./Imdb.java}{Imdb.java}.\\
\par
\noindent
5.\\
Part of the program:\\
Code below would generate a duplicate key error
\begin{lstlisting}
String sql1 ="insert into person values(9993720, 'dongxu', 1997, null);";
String sql2 ="insert into person values(9993720, 'error', 1909, 1992);";
\end{lstlisting}
Code below would catch the error in sql2. And take a rollback on sql1.
\begin{lstlisting}
try {
    stmt.execute(sql1);
    System.out.println("sql1 executed.");
    // sql2 would generate an error and this transaction would rollback
    stmt.execute(sql2);
    stmt.execute(sql3);
    con.commit();
} catch (Exception e) {
    System.out.println("Exception occur: " + e.getMessage());
    if (con != null) {
        con.rollback();
        System.out.println("Rollback occurs.");
    }
}
\end{lstlisting}
Program source code: \href{run:./RollBackTest.java}{RollBackTest.java}.\\

\end{document}